% Options for packages loaded elsewhere
\PassOptionsToPackage{unicode}{hyperref}
\PassOptionsToPackage{hyphens}{url}
%
\documentclass[
  11pt,
]{article}
\author{}
\date{\vspace{-2.5em}}

\usepackage{amsmath,amssymb}
\usepackage{lmodern}
\usepackage{iftex}
\ifPDFTeX
  \usepackage[T1]{fontenc}
  \usepackage[utf8]{inputenc}
  \usepackage{textcomp} % provide euro and other symbols
\else % if luatex or xetex
  \usepackage{unicode-math}
  \defaultfontfeatures{Scale=MatchLowercase}
  \defaultfontfeatures[\rmfamily]{Ligatures=TeX,Scale=1}
\fi
% Use upquote if available, for straight quotes in verbatim environments
\IfFileExists{upquote.sty}{\usepackage{upquote}}{}
\IfFileExists{microtype.sty}{% use microtype if available
  \usepackage[]{microtype}
  \UseMicrotypeSet[protrusion]{basicmath} % disable protrusion for tt fonts
}{}
\makeatletter
\@ifundefined{KOMAClassName}{% if non-KOMA class
  \IfFileExists{parskip.sty}{%
    \usepackage{parskip}
  }{% else
    \setlength{\parindent}{0pt}
    \setlength{\parskip}{6pt plus 2pt minus 1pt}}
}{% if KOMA class
  \KOMAoptions{parskip=half}}
\makeatother
\usepackage{xcolor}
\IfFileExists{xurl.sty}{\usepackage{xurl}}{} % add URL line breaks if available
\IfFileExists{bookmark.sty}{\usepackage{bookmark}}{\usepackage{hyperref}}
\hypersetup{
  hidelinks,
  pdfcreator={LaTeX via pandoc}}
\urlstyle{same} % disable monospaced font for URLs
\usepackage[left = 2.5cm, right = 2cm, top = 2cm, bottom =
2cm]{geometry}
\usepackage{longtable,booktabs,array}
\usepackage{calc} % for calculating minipage widths
% Correct order of tables after \paragraph or \subparagraph
\usepackage{etoolbox}
\makeatletter
\patchcmd\longtable{\par}{\if@noskipsec\mbox{}\fi\par}{}{}
\makeatother
% Allow footnotes in longtable head/foot
\IfFileExists{footnotehyper.sty}{\usepackage{footnotehyper}}{\usepackage{footnote}}
\makesavenoteenv{longtable}
\usepackage{graphicx}
\makeatletter
\def\maxwidth{\ifdim\Gin@nat@width>\linewidth\linewidth\else\Gin@nat@width\fi}
\def\maxheight{\ifdim\Gin@nat@height>\textheight\textheight\else\Gin@nat@height\fi}
\makeatother
% Scale images if necessary, so that they will not overflow the page
% margins by default, and it is still possible to overwrite the defaults
% using explicit options in \includegraphics[width, height, ...]{}
\setkeys{Gin}{width=\maxwidth,height=\maxheight,keepaspectratio}
% Set default figure placement to htbp
\makeatletter
\def\fps@figure{htbp}
\makeatother
\setlength{\emergencystretch}{3em} % prevent overfull lines
\providecommand{\tightlist}{%
  \setlength{\itemsep}{0pt}\setlength{\parskip}{0pt}}
\setcounter{secnumdepth}{5}
\usepackage{newunicodechar}
\usepackage{float}
\usepackage{sectsty}
\usepackage{paralist}
\usepackage{setspace}\spacing{1.5}
\usepackage{fancyhdr}
\usepackage{lastpage}
\usepackage{dcolumn}
\usepackage{natbib}\bibliographystyle{agsm}
\usepackage[nottoc, numbib]{tocbibind}
\usepackage[T1]{fontenc}
\usepackage{palatino}
\newcommand\tab[1][1cm]{\hspace*{#1}}
\newunicodechar{≤}{\ensuremath{\leq}}
\newunicodechar{≥}{\ensuremath{\geq}}
\ifLuaTeX
  \usepackage{selnolig}  % disable illegal ligatures
\fi

\begin{document}

\subsectionfont{\raggedright}
\subsubsectionfont{\raggedright}

\thispagestyle{empty}

\begin{centering}


\begin{center}\includegraphics[width=0.3\linewidth]{ufpi} \end{center}
\begin{singlespace}
\LARGE
UNIVERSIDADE FEDERAL DO PIAUÍ

\Large
CENTRO DE CIÊNCIAS DA NATUREZA

\Large
CURSO DE GRADUAÇÃO EM ESTATÍSTICA
\end{singlespace}

\vspace{3cm}

\LARGE
\doublespacing
{\bf Análise das infrações de trânsito de excesso de velocidade através da Teoria de Valores Extremos}

\vspace{3cm}

\doublespacing

\Large              
{\bf Filipe Mateus de Sousa Costa}

\vspace{3cm}

\doublespacing

\Large  
{\bf TERESINA - 2022}

\end{centering}

\newpage
\thispagestyle{empty}
\begin{centering}

\Large              
{\bf Filipe Mateus de Sousa Costa}

\vspace{5cm}
\doublespacing
\Large              
{\bf Monografia:}


{\bf Análise das infrações de trânsito de excesso de velocidade através da Teoria de Valores Extremos}

\vspace{3cm}
\begin{flushright}
\normalsize
\begin{singlespace}
Monografia submetida à Coordenação do\\
curso de graduação em Estatística, da Uni-\\
versidade  Federal  do Piauí, como requisito\\
parcial  para  obtenção do grau de Bacharel \\
em Estatística.\\
\end{singlespace}
\end{flushright}

\vspace{1cm}
\begin{flushright}
\normalsize
Orientador:\\
Profa. Dra. Valmária Rocha da Silva Ferraz\\
Co-orientador:\\
Prof. Dr. Fernando Ferraz do Nascimento
\end{flushright}


\vspace{2cm}
\Large  
{\bf TERESINA - 2022}

\end{centering}

\newpage
\thispagestyle{empty}
\tableofcontents

\newpage

\listoffigures

\newpage

\listoftables

\newpage

\hypertarget{agradecimentos}{%
\section{Agradecimentos}\label{agradecimentos}}

\hspace*{1cm} Primeiramente, eu preciso agradecer a Deus por ter
permitido a possibilidade de realizar uma segunda graduação e ter me
sustentado durante todo o curso, foram vários os desafios, mas Ele
permitiu que eu conseguisse superar tudo. Soli Deo Glória.\\
\hspace*{1cm} Preciso agradecer a minha esposa que ficou do meu lado
durante este período, sacrificando-se para permitir que eu estudasse,
motivando-me. Superando o ciúme por ter sido trocada, muitas vezes, por
um livro cheio de fórmulas esquisitas. Incluo meu filho Mateus, que
chegou no durante o curso para dar mais emoção à vida. Mesmo pequeno, me
ensinou a ser mais produtivo. E claro, meu pai, minha mãe e minha irmã,
mesmo a distância, me apoiavam.\\
\hspace*{1cm} Agradeço a Professora Valmaria por ter aceitado o desafio
de me orientar nesse TCC, juntamente com o professor Fernando que
ofereceu o apoio teórico para a realização desse projeto. Incluo a
STRANS, na pessoa do superitendente Claudio Pessoa, que gentilmente,
disponibilizaram os dados para este trabalho.\\
\hspace*{1cm} Finalizo os agradecimentos aos amigos que foram criados
durante o curso. Edvaldo que, como eu, chefe de família. Compartilhamos
muito do desafio de trabalhar, cuidar do lar e estudar. Eva, cheia de
energia, se esforçava para tirar as melhores notas nas disciplinas.
Envelheceu alguns anos durante as madrugadas acordadas, mas ok. Obrigado
a todos, de coração.

\newpage

\hypertarget{resumo}{%
\section{Resumo}\label{resumo}}

\newpage

\hypertarget{abstract}{%
\section{Abstract}\label{abstract}}

\newpage

\hypertarget{introduuxe7uxe3o}{%
\section{Introdução}\label{introduuxe7uxe3o}}

\hspace*{1cm} Eventos extremos são situações ou comportamentos que não
ocorrem com tanta frequência. No desenho da distribuição normal, são
aqueles que ficam próximos as caudas, distantes do pico onde está
localizado a média, a situação de normalidade. Entretanto, é possível
que estes eventos ocorram, na verdade, é esperado que eles ocorram.
Diante dessa situação, é importante que possamos prever seus
acontecimentos, pois quando ocorrem, as consequências podem ser
trágicas.

\hspace*{1cm} Exemplos que demonstram com facilidade os efeitos de
eventos extremos estão na área da climatologia. Para os portais de
notícias, o aquecimento global tem trazido mais manchetes de catástrofes
climáticas que causaram prejuízos substanciais e, infelizmente, mortes.
Uma busca rápida em sites de pesquisa nos direciona facilmente para
estes eventos.

\hspace*{1cm} O surgimento da teoria de valores extremos surge da
necessidade de pesquisas sobre eventos extremos de forma mais eficiente.
As distribuições mais comuns trabalham melhor na análise dos eventos
centrais, mais frequentes. Dessa forma, os acontecimentos mais raros
possuem ficam mais difíceis de prever.

\hypertarget{apresentando-os-dados}{%
\subsection{Apresentando os dados}\label{apresentando-os-dados}}

\hspace*{1cm} Os dados que iremos trabalhar neste projeto têm como
origem as medições de excesso de velocidade registradas por radares no
munícipio de Teresina. Baseado no Código de Trânsito Brasileiros, ruas,
avenidas e estradas no Brasil, possuem limites de velocidade
especificados, entretanto, para a realização do registro infração, é
preciso que a via esteja sinalizada informando a velocidade máxima.

\hspace*{1cm} Um ponto importante sobre as infrações de velocidade é que
existe uma margem de erro para medição da velocidade registrada. Para
fim de registro de infração, não se registra a velocidade obtida por
instrumentos de medição, mas subtrai-se um valor na velocidade medida
(para vias de velocidade máxima até 100km/h, o valor é 7), o que passar
deste valor acima da velocidade permitida, será considerado para
infração.

\hspace*{1cm} Por exemplo, para vias de com velocidade máxima de 60
km/h, as infrações serão registradas somente quando o veículo passar
pelo ponto de medição quando a velocidade calculada for de 68 km/h,
registrando um excesso de 1km/h. O resultado dessa distinção é o uso de
dois termos: Velocidade medida, que consiste na velocidade registrada e
Velocidade Considerada, que é a velocidade medida menos o fator de
correção. Na tabela abaixo, temos a relação entre velocidade medida e
velocidade considerada. A infração é feita quando a considerada é maior
que permitida na via.

\begin{figure}[H]

{\centering \includegraphics[height=0.93\textheight]{1_MINFRA_9_002} 

}

\caption{Tabela com relação Velocidade Medida (VM) e Velocidade Considerada (VC)}\label{fig:unnamed-chunk-1}
\end{figure}

\hspace*{1cm} Na tabela acima, temos os valores entres as velocidades
medidas e consideradas, note-se que na medida 107 e 108 km/h, a
considerada se repete, neste ponto se inicia um aumento da margem de
erro. Para este trabalho, iremos utilizar a medição calculada com a
Velocidade considerada, pois o cálculo da infração é realizado com ela.

\hspace*{1cm} Em relação as infrações, o CTB normatiza três tipos de
infrações para o excesso de velocidade baseado no percentual registado
acima. Para medições até 20\% acima da permitida, a infração é média,
entre 20\% e 50\% é considerada grave e, superior a 50\%, é gravíssima.
O aumento da gravidade resulta no aumento do valor pago da multa. Nas
tabelas abaixo, temos um resumo da divisão de tipificação, o intervalo
de excesso e o valor pago. A primeira tabela refere-se para vias com
velocidade máxima até 60 km/h e a segunda tabela considera a máxima
40km/h

\begin{itemize}
\tightlist
\item
  \textbf{Radares de 60 km/h}
\end{itemize}

\begin{longtable}[]{@{}llll@{}}
\toprule
Tipo & Faixa percentual & Faixa de Excesso & Valor da Multa \\
\midrule
\endhead
Média & até 20\% & 1 ≤ v ≤ 12 & R\$ 130,16 \\
Grave & entre 20\% e 50\% & 13 ≤ v ≤ 30 & R\$ 195,23 \\
Gravíssima & superior a 50\% & ≥ 31 & R\$ 880,41 \\
\bottomrule
\end{longtable}

\begin{itemize}
\tightlist
\item
  \textbf{Radares de 40 km/h}
\end{itemize}

\begin{longtable}[]{@{}llll@{}}
\toprule
Tipo & Faixa percentual & Faixa de Excesso & Valor da Multa \\
\midrule
\endhead
Média & até 20\% & 1 ≤ v ≤ 8 & R\$ 130,16 \\
Grave & entre 20\% e 50\% & 9 ≤ v ≤ 20 & R\$ 195,23 \\
Gravíssima & superior a 50\% & ≥ 21 & R\$ 880,41 \\
\bottomrule
\end{longtable}

\hypertarget{pontos-de-mediuxe7uxe3o-de-excesso-de-velocidade}{%
\subsubsection{Pontos de medição de excesso de
velocidade}\label{pontos-de-mediuxe7uxe3o-de-excesso-de-velocidade}}

\hspace*{1cm} Para a realização deste trabalho, escolhemos 4 endereços,
são eles:

\begin{itemize}
\item
  Alameda Parnaíba, próximo Ponte Estaiada João Isidoro França -- Zona
  Norte;
\item
  Av. Raul Lopes, em frente ao Teresina Shopping -- Zona Leste;
\item
  Av. Maranhão, trecho entre o centro Administrativo e ponte da Amizade
  -- Zona Sul;
\item
  Av. Barão de Castelo Branco, próximo Igreja Católica do Cristo Rei. --
  Zona Sul.
\end{itemize}

\hspace*{1cm} Estes radares são considerados famosos com velocidade
máxima de 60 km/h, exceção do radar da Av. Barão de Castelo Branco, onde
o excesso é de 40 km/h. Seus bancos de dados são extensos e iremos
apresentar algumas medidas descritivas para termos uma noção melhor da
distribuição dos excessos.

\hspace*{1cm} É importante destacar que não iremos realizar uma
distinção do sentido que o veículo está indo, para análises futuras,
pode-se aprofundar nesta análise. Destaco, também, que utilizaremos
todas os registros realizados, independente se auto não foi expedido por
motivos técnicos ou administrativos.

\hypertarget{detalhando-os-dados}{%
\subsubsection{Detalhando os dados}\label{detalhando-os-dados}}

\hspace*{1cm} Os dados utilizados apresentam algumas diferenças de
período de registro por via, entretanto, todos os dados têm a data
limite de 31/12/2021, sendo seus inícios variados. Isto ocorre devido a
fatores administrativos e técnicos. Um fator que não podemos mensurar
facilmente são os dias zerados para autuações, pois não podemos
identificar se foi problema técnico ou se realmente não houve nenhum
registro naquele específico.

\hspace*{1cm} Abaixo, apresentamos uma tabela especificando a data de
início e fim dos registros, a quantidade dias entre as datas e a
quantidade de dias zerados.

\begin{longtable}[]{@{}lllrr@{}}
\toprule
Endereço & Data de Inicio & Data Final & Dias & Dias Zerados \\
\midrule
\endhead
Alameda Parnaíba & 05/05/2016 & 31/12/2021 & 2089 & 282 \\
Av. Raul Lopes & 12/05/2017 & 19/10/2020 & 1696 & 595 \\
Av. Maranhão & 05/05/2017 & 31/12/2021 & 1702 & 292 \\
Av. B. C. Branco & 26/09/2016 & 31/12/2020 & 1830 & 334 \\
\bottomrule
\end{longtable}

\hspace*{1cm} A próxima tabela apresenta a quantidade de dias com
registros de autuação, a quantidade de autos registrados em todo o
período analisando e a divisão percentual por tipo de infração
apresentada anteriomente, média, grave e gravissima.

\begin{longtable}[]{@{}
  >{\raggedright\arraybackslash}p{(\columnwidth - 10\tabcolsep) * \real{0.23}}
  >{\raggedleft\arraybackslash}p{(\columnwidth - 10\tabcolsep) * \real{0.05}}
  >{\raggedleft\arraybackslash}p{(\columnwidth - 10\tabcolsep) * \real{0.11}}
  >{\raggedleft\arraybackslash}p{(\columnwidth - 10\tabcolsep) * \real{0.17}}
  >{\raggedleft\arraybackslash}p{(\columnwidth - 10\tabcolsep) * \real{0.25}}
  >{\raggedleft\arraybackslash}p{(\columnwidth - 10\tabcolsep) * \real{0.19}}@{}}
\toprule
\begin{minipage}[b]{\linewidth}\raggedright
Endereço
\end{minipage} & \begin{minipage}[b]{\linewidth}\raggedleft
Dias
\end{minipage} & \begin{minipage}[b]{\linewidth}\raggedleft
Nº de Autos
\end{minipage} & \begin{minipage}[b]{\linewidth}\raggedleft
Percentual até 20\%
\end{minipage} & \begin{minipage}[b]{\linewidth}\raggedleft
Percentual entre 20\% e 50\%
\end{minipage} & \begin{minipage}[b]{\linewidth}\raggedleft
Percentual acima 50\%
\end{minipage} \\
\midrule
\endhead
Alameda Parnaiba & 1807 & 48647 & 90.23 & 9.30 & 0.45 \\
Av. Raul Lopes & 1101 & 51605 & 92.75 & 7.03 & 0.22 \\
Av. Maranhão & 1410 & 27273 & 88.34 & 10.94 & 0.72 \\
Av. B. de Castelo Branco & 1496 & 30954 & 79.18 & 17.66 & 3.16 \\
\bottomrule
\end{longtable}

\hypertarget{anuxe1lise-descritiva-e-da-distribuiuxe7uxe3o-dos-excessos.}{%
\subsubsection{Análise descritiva e da distribuição dos
excessos.}\label{anuxe1lise-descritiva-e-da-distribuiuxe7uxe3o-dos-excessos.}}

\hspace*{1cm} O objetivo deste tópico é apresentar algumas medidas
básicas dos radares que estamos analisando. Primeiramente, temos um
tabela com as informações descritivas dos radares. Num segundo momento,
construímos algumas imagens com \texttt{ggplot} com as distribuições dos
excessos dos radares.

\begin{longtable}[]{@{}lrrrrr@{}}
\toprule
Endereço & Média & Desvio Padrão & Mediana & Mínimo & Máximo \\
\midrule
\endhead
Alameda Parnaiba & 5.61 & 5.39 & 4 & 1 & 67 \\
Av. Raul Lopes & 5.02 & 5.02 & 4 & 1 & 92 \\
Av. Maranhão & 6.05 & 5.94 & 4 & 1 & 70 \\
Av. B. de Castelo Branco & 5.74 & 5.89 & 4 & 1 & 109 \\
\bottomrule
\end{longtable}

\begin{figure}

{\centering \includegraphics[width=432px]{versaoFinalTCC_files/figure-latex/unnamed-chunk-8-1} 

}

\caption{Distribuição de excesso na Alameda Parnaíba}\label{fig:unnamed-chunk-8}
\end{figure}

\begin{figure}

{\centering \includegraphics[width=432px]{versaoFinalTCC_files/figure-latex/fig.-1} 

}

\caption{Distribuição de excesso na Av. Raul Lopes}\label{fig:fig.}
\end{figure}

\begin{figure}

{\centering \includegraphics[width=432px]{versaoFinalTCC_files/figure-latex/unnamed-chunk-9-1} 

}

\caption{Distribuição de excesso na Av. Maranhão}\label{fig:unnamed-chunk-9}
\end{figure}

\begin{figure}

{\centering \includegraphics[width=432px]{versaoFinalTCC_files/figure-latex/unnamed-chunk-10-1} 

}

\caption{Distribuição de excesso na Av. B. Castelo Branco}\label{fig:unnamed-chunk-10}
\end{figure}

\newpage

\hypertarget{alguns-testes}{%
\subsubsection{Alguns testes}\label{alguns-testes}}

\hypertarget{teste-de-corrridas}{%
\paragraph{Teste de Corrridas}\label{teste-de-corrridas}}

\begin{itemize}
\tightlist
\item
  \textbf{Alameda Parnaíba}
\end{itemize}

\begin{verbatim}
## 
##  Runs Test for Randomness
## 
## data:  Alameda$Excesso
## z = -2.8499, runs = 23806, m = 26564, n = 22083, p-value = 0.004373
## alternative hypothesis: true number of runs is not equal the expected number
## sample estimates:
## median(x) 
##         4
\end{verbatim}

\begin{itemize}
\tightlist
\item
  \textbf{Av. Raul Lopes}
\end{itemize}

\begin{verbatim}
## 
##  Runs Test for Randomness
## 
## data:  Shopping$Excesso
## z = -2.6169, runs = 24539, m = 30826, n = 20779, p-value = 0.008872
## alternative hypothesis: true number of runs is not equal the expected number
## sample estimates:
## median(x) 
##         4
\end{verbatim}

\begin{itemize}
\tightlist
\item
  \textbf{Av. Maranhão}
\end{itemize}

\begin{verbatim}
## 
##  Runs Test for Randomness
## 
## data:  Maranhao$Excesso
## z = -3.1474, runs = 13355, m = 14191, n = 13082, p-value = 0.001647
## alternative hypothesis: true number of runs is not equal the expected number
## sample estimates:
## median(x) 
##         4
\end{verbatim}

\begin{itemize}
\tightlist
\item
  \textbf{Av. B. de Castelo Branco}
\end{itemize}

\begin{verbatim}
## 
##  Runs Test for Randomness
## 
## data:  Barao$Excesso
## z = -2.1968, runs = 15009, m = 17554, n = 13400, p-value = 0.02803
## alternative hypothesis: true number of runs is not equal the expected number
## sample estimates:
## median(x) 
##         4
\end{verbatim}

\hypertarget{teste-de-kruskal-white}{%
\paragraph{Teste de Kruskal White}\label{teste-de-kruskal-white}}

\begin{verbatim}
## # A tibble: 1 x 4
##   statistic  p.value parameter method                      
##       <dbl>    <dbl>     <int> <chr>                       
## 1      435. 5.45e-94         3 Kruskal-Wallis rank sum test
\end{verbatim}

\hypertarget{puxf3s-teste}{%
\paragraph{Pós-Teste}\label{puxf3s-teste}}

\begin{longtable}[]{@{}lllrrrrrl@{}}
\toprule
.y. & group1 & group2 & n1 & n2 & statistic & p & p.adj &
p.adj.signif \\
\midrule
\endhead
excessos & 1 & 2 & 48647 & 51605 & -13.916882 & 0.0000000 & 0 & **** \\
excessos & 1 & 3 & 48647 & 27273 & 7.843274 & 0.0000000 & 0 & **** \\
excessos & 1 & 4 & 48647 & 30954 & -0.867789 & 0.3855098 & 1 & ns \\
excessos & 2 & 3 & 51605 & 27273 & 19.672889 & 0.0000000 & 0 & **** \\
excessos & 2 & 4 & 51605 & 30954 & 11.355477 & 0.0000000 & 0 & **** \\
excessos & 3 & 4 & 27273 & 30954 & -7.903749 & 0.0000000 & 0 & **** \\
\bottomrule
\end{longtable}

\begin{itemize}
\tightlist
\item
  \textbf{Gráfico comparativo}
\end{itemize}

\begin{figure}

{\centering \includegraphics[width=432px]{versaoFinalTCC_files/figure-latex/unnamed-chunk-17-1} 

}

\caption{Boxplot comparativo das distribuições dos excessos}\label{fig:unnamed-chunk-17}
\end{figure}

\hypertarget{arrecadauxe7uxe3o-possuxedvel}{%
\subsubsection{Arrecadação
possível}\label{arrecadauxe7uxe3o-possuxedvel}}

\hspace*{1cm} Finalizando esta análise no ponto que consiste o objetivo
final do trabalho, a arrecadação. A idéia do trabalho é utilizar as
informações detalhadas acima para que seja possível realizar uma
previsão de arrecadação e, desta forma, aperfeiçoar a forma de
administração das receitas e despesas públicas. Abaixo, temos a
arredacação possível se todas as multas fosses pagas com o valor
integral.

\begin{longtable}[]{@{}ll@{}}
\toprule
Local & Arrecadação Possível \\
\midrule
\endhead
Alameda Parnaiba & R\$ 6.794.927 \\
Av. Raul Lopes & R\$ 7.036.259 \\
Av. Maranhão & R\$ 3.891.757 \\
Av. B. de Castelo Branco & R\$ 5.119.140 \\
\bottomrule
\end{longtable}

\newpage

\hypertarget{distribuiuxe7uxe3o-pareto-generalizada-gpd}{%
\section{Distribuição Pareto Generalizada
(GPD)}\label{distribuiuxe7uxe3o-pareto-generalizada-gpd}}

\hspace*{1cm} A distribuição Pareto Generalizada (GPD) analisa a
distribuição dos excessos de acordo com um limiar determinado. Esse
formato de análise é mais eficaz pois evita a perda de informações que
uma análise em períodos (ou blocos) pode gerar, afetando,
principalmente, pesquisas com um grande volume de dados.

\hspace*{1cm} A distribuição Pareto Generalizada foi desenvolvida por
Pickands {[}1975{]} é baseada no seguinte teorema:

\emph{Teorema 1: Se x for uma variável aleatória (v.a.) com função
distribuição (f.d.) \(F_{x}\), que pertence ao domínio da de atração de
uma distribuição GEV, então, quando \(\mu \to \infty\),
\(F(x|u) = Pr{X > u + x|X > u}\), possui distribuição GPD, com a
seguinte função de distribuição:}

\[
P(x|\xi, \sigma, \mu)\ = \
\left\{ \begin{array}{rcl}
1 - (1 + \xi\frac{(x - \mu)}{\sigma})^{-\frac{1}{\xi}},\ \mbox{se}\ \xi \neq 0\\
1 - exp\left\{-\frac{(x - \mu)}{\sigma}\right\}, \ \mbox{se}\ \xi = 0
\end{array}\right.
\]

onde \(\mu > 0, x - \mu \geqslant 0\), se \(\xi > 0\), e
\(0 \leqslant x - \mu \leqslant - \frac{\sigma}{\xi}\), se \(\xi < 0\).
O caso \(\xi = 0\) é interpretado como sendo o limite quando
\(\xi \rightarrow 0\), e tem como caso particular a
distribuiçãoexponencial de parâmetro \(\frac{1}{\sigma}\). Os parametros
são \(\xi\), \(\sigma\) e \(\mu\), que representam, respectivamente, a
forma, a escala e o limiar da distribuição.

\hspace*{1cm} A função de densidade da distribuição GPD é dada por:

\[
p(x|\xi, \sigma, \mu)\ = \
\left\{ \begin{array}{rcl}
\frac{1}{\sigma}(1 + \xi\frac{x - \mu}{\sigma})^{-\frac{1}{\xi}},\ \mbox{se}& \xi \neq 0\\
\frac{1}{\sigma}exp\left\{-\frac{(x - \mu)}{\sigma}\right\}, \ \mbox{se}& \xi = 0
\end{array}\right.
\]

onde \(x - \mu > 0\) para \(\xi \geqslant 0\) e
\(0 \leqslant x - \mu < -\frac{\mu}{\xi}\) para \(\xi > 0\).

\hspace*{1cm} A distribuição GPD possui as seguintes características em
relação aos parâmetros:

\[
E(X)\ =\ \frac{\sigma}{1 - \xi};\ 
\xi < 1; Md(x)\ =\ \frac{\sigma(2\xi -1)}{\xi};
V(X)\ =\ \frac{\sigma^{2}}{(1-\xi)^{2}(1-2\xi)}
\]

\hspace*{1cm} Justificando o uso da GPD Pickands (1975) e Davidson e
Smith (1990) demostram as propriedades e provam que GPD é única que
satisfaz estas propriedades. ``Por exemplo, estabilidade do limiar, ou
seja, se Y possui distribuição GPD, e se \(\mu > 0\) , então a
distribuição de \(P(Y - \mu|Y>\mu)\) também possui distribuição GPD''.
\emph{prof fernando}

\hspace*{1cm} Em valores extremos, além de encontrar a estimativa dos
parâmetros do modelo, também é muito importante encontrar uma forma para
determinar os quatis altos, acima do limiar, de tal forma que se X
possui distribuição GPD, é importante saber com qual probabilidade
ocorre um evento maior ou igual a q, ou seja, \(P(X>q) = 1- q\).

\hspace*{1cm} Com os cálculos destes quantis, podemos realizar previsões
com os dados de autos de excesso velocidade de trânsito de Teresina nos
próximos anos, considerando uma manutenção da estrutura dos radares e,
incluindo novos endereços de medição, uma previsão de quantos autos
poderão ser registrados. Outra variável que pode ser analisada consiste
no excessos de velocidade, determinando possíveis valores máximos e a
previsão da quantidade de infrações.

\hspace*{1cm} Na distribuição GPD, pode-se encontrar um quantil q com
probabilidade \(P(X < q)\) em função dos parâmetros. Invertendo a função
acumulada, obtém-se a seguinte função dos quantis da cauda:

\[
q_{x}p = \frac{((1 - p*)^{-\xi}-1)}{\xi},
\] onde \(p*\ =\ 1 - (1-p)N/N_{u}\).

\hypertarget{domuxednio-da-atrauxe7uxe3o}{%
\subsection{Domínio da atração}\label{domuxednio-da-atrauxe7uxe3o}}

\hspace*{1cm} As distribuições de valores extremos são obtidas como
distribuição limite \((n \rightarrow \infty)\) do máximo de um conjunto
de variáveis aleatórias (v.a.s) independente e identicamente
distribuídas (i.i.d) e são unicamente determinadas, a menos de
transformac˜oes afins. O teorema de Fisher-Tippet implica que se
\(F^{n}_{x}(C_{n}x + d_{n})\) e n˜ao degenerada quando
\((n \rightarrow \infty)\), para certas constantes
\(C_{n} > 0,\ d_{n}\ \varepsilon\ \mathbb{R}\), então

\[
|F^{n}_{X}(x)\ -\ H(\frac{x - d_{n}}{c_{n}})|\  \rightarrow 0, n \rightarrow \infty
\]

para alguma distribuição H. A coleção das distribuições \(F_{x}\) tais
que os respectivos máximos possuem a mesma distribuição limite é chamada
de domínio de atração.

\emph{Definição 2.1.2 Se (função) se verifica dizemos que} \(F_{x}\)
pertence ao domínio de atração do máximo da distribuição de valores
extremos H. Notação: \(F_{x}\  \epsilon\ MDA(H)\)

\hspace*{1cm} Existem 3 casos possíveis para as distribuições limites
das excedências de um limiar. Para domínio dp tipo I \((\gamma = 0)\), a
distribuição se torna

\[
H(y)\ = 1 - e^{-\frac{y}{\sigma}},\ y > 0
\]

sendo assim, o domínio, uma distribuição Exponencial com parâmetro
\(\frac{1}{\sigma}\). Para o domínio tipo II \((\gamma > 0)\), a
distribuição limite será a distribuição de Pareto . Já para o domínio
tipo III \((\gamma < 0)\), quando \(\sigma = -\frac{1}{\gamma}\) , a
distribuição limite será uma Beta e quando
\(\sigma \neq -\frac{1}{\gamma}\), adistribuição limite será uma Beta
reescalada com suporte em \((0,\frac{\sigma}{\gamma})\).

\hypertarget{determinauxe7uxe3o-do-limiar}{%
\subsection{Determinação do Limiar}\label{determinauxe7uxe3o-do-limiar}}

\hspace*{1cm} A análise via GPD exige um cuidado inicial pois é preciso
determinar um limiar para os dados. O valor escolhido pode ser
determinado pelo pesquisador, entretanto, correm-se riscos que podem
influenciar os cálculos, resultando em análises incorretas.

\hspace*{1cm} A escolha de um limiar \("\mu"\) muito alto implica em um
número muito pequeno de observações resultando em estimadores com grande
variabilidade. Um limiar muito pequeno resulta na violação do Teorema de
Pickands (1945), modelando de forma errada os valores com limiar baixo,
dessa forma, não se garante a convergência dos excessos Y para a família
da GPD, levando a um vício alto.

\hspace*{1cm} Métodos mais convencionais de determinação do limiar
utilizam-se de análises gráficas da linearidade de \(N_{u}\). Um método
muito utilizado é o gráfico de médias de excessos (MRL\textless{}
\emph{Mena Residual Life Plot}), baseado na espera da GPD (Nascimento
{[}2012{]}). Sua construção segue o seguinte formato:

\[
\left\{\left(\mu,\ \frac{1}{n_{u}}\sum_{i=1}^{n_{u}} \right): \mu<x_{max}\right\} 
\]

onde \(x_{1}\leqslant x_{2}\leqslant...x_{n}\) consistem nas \(N_{u}\)
observac˜oes que excedem \(\mu\), e \(x_{max}\) é o valor mais elevado
das obervações.

\hspace*{1cm} Considerando a distribuição GPD válida para os excesso,
esta também é valida para os excesso acima de todos os limiares
\(\mu > \mu_{0}\) 0, sujeito a mudan¸cas no parâmetro de escala
\(\sigma_{\mu} = \sigma_{\mu_{0}}+\xi_{\mu}\). Então, para
\(\mu > \mu_{o}\)

\[
E(X - \mu|X>\mu)= \frac{\sigma_{\mu}}{1 -\xi}=\frac{\sigma_{\mu_{0}}+\xi_{\mu}}{1 - \xi}
\] Se o modelo é adequado a partir de \(\mu_{0}\) o gráfico apresentará
um comportamento linear a partir de u. Um problema recorrente com a
utilização desse gráfico é que o limiar pode limitar o número de
excessos devido a escolha de limiar muito alto.

\hspace*{1cm} Outra técnica gráfica utilizada é Dipersion Index Plot
(DIP), baseado em Cunnane {[}1979{]} (Citado por Nascimento {[}2012{]}),
que diz que, o número de excessos sobre um limiar alto em um determinado
período (geralmente meses ou anos), pode ser distribuído através de um
processo de Poisson. Assim, a razão entre a variância e a média é igual
a 1. Assim, pode-se fazer um gráfico

\[
\left\{\left(\mu, \frac{Var(Y)}{E(Y)}\right): \mu <x_{max}\right\}
\]

\hypertarget{estimauxe7uxe3o-da-gpd}{%
\subsection{Estimação da GPD}\label{estimauxe7uxe3o-da-gpd}}

\hspace*{1cm} Após determinar o limiar, a estimação dos parâmetros da
distribuição GPD podem ser estimados por vários métodos, entre eles,
tem-se o da máxima verossimilhança, que iremos trabalhar. Existem outros
métodos como de momentos proposto por Smith {[}1987{]} (citado por
Mendes {[}2004{]}) e o métodos dos momentos ponderados (Singh e Guo
{[}1995{]}, (citado por Mendes {[}2004{]}), em que a eficiência de cada
método depende da situação estudada.

\hspace*{1cm} Os estimadores de máxima verossimilhança (EMV) que
maximizam a função de log-verossimilhança, quando \(\xi \neq 0\), é dado
por

\[
l(\mu, \xi)\ =\ -n_{u}log(\sigma) - (1+\frac{1}{\xi}\sum_{i=1}^{n_{u}}log(1+\xi\frac{y_{i}}{\sigma}))
\] definida em \((1 + \xi\frac{y_{i}}{\sigma})>0\), para todo
\(i = 1,2,...,N_{u}\). No caso particular onde \(\xi = 0\), a a
log-verossimilhança é dada por

\[
l(\sigma)\ =\ -n_{u}log(\sigma)\ -\ \sum_{i=1}^{n_{u}}(\frac{y_i}{\sigma})
\]

\hspace*{1cm} No caso de \(\xi = 0\), a maximização dos parâmetros não
pode ser obtida analiticamente, sendo necessárias técnicas numéricas de
maximização.

\newpage

\newpage

\hypertarget{inferuxeancia-bayesiana}{%
\section{Inferência Bayesiana}\label{inferuxeancia-bayesiana}}

\newpage

\hypertarget{modelos-dinuxe2micos-para-valores-extremos}{%
\section{Modelos dinâmicos para Valores
Extremos}\label{modelos-dinuxe2micos-para-valores-extremos}}

\hypertarget{alameda-parnauxedba}{%
\subsection{Alameda Parnaíba}\label{alameda-parnauxedba}}

\begin{figure}

{\centering \includegraphics[width=350px]{Alameda1} 

}

\caption{Alameda Parnaíba - Função fggpd}\label{fig:unnamed-chunk-19}
\end{figure}

\begin{figure}

{\centering \includegraphics[width=350px]{Alameda2} 

}

\caption{Alameda Parnaíba - Função fmgpd}\label{fig:unnamed-chunk-20}
\end{figure}

\begin{longtable}[]{@{}lrr@{}}
\toprule
Ajuste & DIC & BIC \\
\midrule
\endhead
Alameda com fggpd & 247310.8 & 247351.3 \\
Alameda com fmgpd & 215577.4 & 215641.7 \\
\bottomrule
\end{longtable}

\hypertarget{av.-maranhuxe3o}{%
\subsection{Av. Maranhão}\label{av.-maranhuxe3o}}

\begin{figure}

{\centering \includegraphics[width=350px]{Maranhao1} 

}

\caption{Av. Maranhão - Função fggpd}\label{fig:unnamed-chunk-22}
\end{figure}
\begin{figure}

{\centering \includegraphics[width=350px]{Maranhao2} 

}

\caption{Av. Maranhão - Função fmgpd}\label{fig:unnamed-chunk-23}
\end{figure}

\begin{longtable}[]{@{}lrr@{}}
\toprule
Ajuste & DIC & BIC \\
\midrule
\endhead
Av. Maranhao com fggpd & 480153.4 & 480201.7 \\
Av. Maranhão com fmgpd & 485451.3 & 485518.7 \\
\bottomrule
\end{longtable}

\hypertarget{av.-raul-lopes---shopping}{%
\subsection{Av. Raul Lopes - Shopping}\label{av.-raul-lopes---shopping}}

\begin{figure}

{\centering \includegraphics[width=350px]{Shopping1} 

}

\caption{Av. Raul Lopes - Função fggpd}\label{fig:unnamed-chunk-25}
\end{figure}

\begin{figure}

{\centering \includegraphics[width=350px]{Shopping2} 

}

\caption{Av. Raul Lopes - Função fmgpd}\label{fig:unnamed-chunk-26}
\end{figure}

\begin{longtable}[]{@{}lrr@{}}
\toprule
Ajuste & DIC & BIC \\
\midrule
\endhead
Av. Raul Lopes com fggpd & 253692.7 & 253754.8 \\
Av. Rul Lopes com fmgpd & 211497.7 & 211567.5 \\
\bottomrule
\end{longtable}

\hypertarget{av.-dos-ipuxeas}{%
\subsection{Av. dos Ipês}\label{av.-dos-ipuxeas}}

\begin{figure}

{\centering \includegraphics[width=350px]{Ipes1} 

}

\caption{Av. dos Ipês - Função fggpd}\label{fig:unnamed-chunk-28}
\end{figure}

\begin{figure}

{\centering \includegraphics[width=350px]{Ipes2} 

}

\caption{Av. dos Ipês - Função fmgpd}\label{fig:unnamed-chunk-29}
\end{figure}

\begin{longtable}[]{@{}lrr@{}}
\toprule
Ajuste & DIC & BIC \\
\midrule
\endhead
Av. Av. dos Ipês com fggpd & 136611.9 & 136654.4 \\
Av. dos Ipês com fmgpd & 131887.8 & 131672.3 \\
\bottomrule
\end{longtable}

\hypertarget{av.-av.-jose-francisco-de-almeida-neto}{%
\subsection{Av. Av. Jose Francisco de Almeida
Neto}\label{av.-av.-jose-francisco-de-almeida-neto}}

\begin{figure}

{\centering \includegraphics[width=350px]{Jose1} 

}

\caption{Av. Jose - Função fggpd}\label{fig:unnamed-chunk-31}
\end{figure}

\begin{figure}

{\centering \includegraphics[width=350px]{Jose2} 

}

\caption{Av. Jose - Função fmgpd}\label{fig:unnamed-chunk-32}
\end{figure}

\begin{longtable}[]{@{}lrr@{}}
\toprule
Ajuste & DIC & BIC \\
\midrule
\endhead
Av. Av. Jose com fggpd & 313760.1 & 303626.5 \\
Av. Jose com fmgpd & 313806.8 & 303289.6 \\
\bottomrule
\end{longtable}

\hypertarget{av.-baruxe3o-de-castelo-branco}{%
\subsection{Av. Barão de Castelo
Branco}\label{av.-baruxe3o-de-castelo-branco}}

\begin{figure}

{\centering \includegraphics[width=350px]{Barao1} 

}

\caption{Av. Barão - Função fggpd}\label{fig:unnamed-chunk-34}
\end{figure}
\begin{figure}

{\centering \includegraphics[width=350px]{Barao2} 

}

\caption{Av. Barão - Função fmgpd}\label{fig:unnamed-chunk-35}
\end{figure}

\begin{longtable}[]{@{}lrr@{}}
\toprule
Ajuste & DIC & BIC \\
\midrule
\endhead
Av. Barão com fggpd & 160774.3 & 160813.9 \\
Av. Barão com fmgpd & 153049.2 & 152921.9 \\
\bottomrule
\end{longtable}

\newpage

\hypertarget{conclusuxe3o}{%
\section{Conclusão}\label{conclusuxe3o}}

\newpage

\hypertarget{referuxeancias-bibliogruxe1ficas}{%
\section{Referências
Bibliográficas}\label{referuxeancias-bibliogruxe1ficas}}

@book\{AREL:03, author=\{Nascimento, F. F.\}, title=\{Modelos
probabilísticos para dados extremos: teoria e aplicações\},
publisher=\{EDUFPI\}, year=\{2012.\} \}

\end{document}
